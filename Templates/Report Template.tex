\documentclass[letterpaper, 12pt]{article}
\usepackage[utf8]{inputenc}
\usepackage{amsmath}
\usepackage{fullpage}
\usepackage{indentfirst}
\usepackage{graphicx}
\usepackage{color}
%Setup standard date using \today command:
\renewcommand{\today}{\number\day\space\ifcase\month\or
   January\or February\or March\or April\or May\or June\or
   July\or August\or September\or October\or November\or December\fi
   \space\number\year}
%Vector formatting:
\newcommand{\vect}[1]{\boldsymbol{#1}}

\title{Lab Title}
\author{Author}
\date{Department of Physics, Case Western Reserve University \\ Cleveland, OH 44106-7079}

\begin{document}
\maketitle

	\subsection*{Abstract:}
	
		Give a brief description of what we were measuring, how those measurements were made, the concluding data, and the basic conclusion drawn from that data. Write this LAST so I can finish the analysis
	
	\subsection*{Introduction and Theory}
	
		Motivate investigations and show how data will be analyzed.
	
	\subsection*{Experimental Procedure}
	
		Explain the experimental technique in complete sentences within paragraphs. It is important to include diagrams of the equipment setup. Describe how data is measured in the experiment and any instruments used to do so. Note the measurement error associated with each value.
	
	\subsection*{Results and Analysis}
	
		Show data and mathematical analysis of that data. It may help to include some of the analysis in this section, before the conclusion. Avoid tables or lists of data; when possible, create graphs or charts that communicate the data. Include comments about how the analysis proceeds and appropriately reference prior equations as necessary. Demonstrate only one calculation series, even if it was applied to a large set of data. Make sure to show enough work that the calculation is clear. 
		
		Include information on error propagation, as necessary. Extensive calculations may be moved to an appendix. 
	
	\subsection*{Conclusions}
	
		Compare the experimental and expected results, considering both the value itself and the error estimate. Briefly restate the goals of the experiment and summarize the experimental outcome in relation to the goals. Consider writing out possible explanations for anything incorrect, but do not express confidence in these explanations. Draw connections to existing works and discuss how any problems encountered in the experiment could be solved.
		
		The important elements of this section should appear in the Abstract.
	
	\subsection*{Acknowledgements}
	
		Rather than extensively using footnotes and/or references, this section allows me to acknowledge any people or organizations that significantly contributed to the process of this experiment.
	
	\subsection*{References}
	\begin{enumerate}
		\item Lab Manual Reference
	\end{enumerate}

\end{document}